\documentclass{article}

\usepackage{tikz}
\usepackage{graphicx}
\usetikzlibrary{shapes,arrows}
\usepackage[utf8]{inputenc}
\usepackage[english]{babel}
\usepackage{fancyhdr}
 
\pagestyle{fancy}
\lhead{}
\chead{}
\rhead{}
\rfoot{\thepage}
\cfoot{Software User manual 0.0.0}
 \lfoot{Umass Amherst}
\renewcommand{\headrulewidth}{2pt}
\renewcommand{\footrulewidth}{1pt}
 
\usepackage{setspace}
\title{\Huge{Oracle-Guided Incremental SAT Solver User manual}}
\author{\small{Duo Liu, Cunxi Yu, Xiangyu Zhang, Daniel Holcomb}}
\date{}
\linespread{1.5}

\begin{document}

\maketitle
\centerline{\Large{Software User Manual}}
\vspace{10pt}
\centerline{\small{\emph{version 0.0.0, 04 Nov 2015}}}


\newpage
\section*{Abstract} 
	This document is the Software User Manual (SUM) for the Oracle-Guided Incremental SAT Solver (Solver). The Software User Manual (SUM) instructs how to install and use the Oracle-Guided Incremental SAT Solver (Solver). 
	
	
\newpage
\tableofcontents
\clearpage
\section{Introduction}
	\subsection{Purpose}
	\paragraph{} Solving camouflage circuit is a notoriously NP problem. The Oracle-Guided Incremental SAT Solver (Solver) is specified in solving camouflage circuit with extremely high efficiency (5 times faster than existing best solver).  		
	\subsection{Principle} 
	\paragraph{}Solver executes a loop that continually finds new input and output vectors using SAT queries and an oracle circuit model. After some number of iterations, the constraints accumulated are sufficient to rule out all logical functions except for the one that is the true function of the obfuscated circuit. Detailed description please refer to citation. 
	\subsection{Terminology}
		\begin{itemize}
			\item Oracle circuit: original circuit without any obfuscated gate.
			\item Camouflage circuit: obfuscated $oracle$ circuit.
			\item allowed bits: possible solution, known in advance, for camouflage circuit
			\item forbidden bits: complementary set of allowed bits
		\end{itemize}

\section{Tutorial}
	\subsection{Dependencies}
		\begin{itemize}
			\item minisat-incre-simp: main program.
			\item decam-incre.py, genMtrs.py, grabNodes.py: parser, translate combinational circuits to CNF clauses by way of Tseitin encoding
			\item completeset.py, testforbid.py: library, used to assign forbidden bits
		\end{itemize}
	\subsection{Initialization}
		Makefile is included in directory, use command below to initialize working environment.
		\newline 
		\centerline{\$ make}
	\subsection{Command}
		After initializing, $solver$ can be accessed from command line:
		\newline
		\centerline{ \$ ./minisat-incre-simp decam-incre $<$oracle$.v>$ $<$camouflage$.v>$}
			\begin{itemize}
				\item $oracle$.v : input $oracle$ circuit path 
				\item $camouflage$.v : input $camouflage$ circuit path
			\end{itemize}
		For example, if the oracle circuit is "c432-abcmap-fmt.v" and the camouflage circuit is "c432-mux4-101.v", then the command should be:
		\newline
		\centerline{\$ ./minisat-incre-simp decam-incre c432-abcmap-fmt.v c432-mux4-101.v}
	\subsection{Sample input format}
	$Solver$ only takes ABC format netlist, the following is an example: 
	\newline
	\begin{tabular}{|p{13cm}|}
 		 \begin{verbatim}
module	c17 (N1,N2,N3,N4,N5,N6,N7,N8,N9,N12,N13,N14);
input N1,N2_new,N3,N4,N5  //RE__PI;
input X_1 //RE__ALLOW(0,1);
input p1,p2,p3,p4     //RE__ALLOW(1,0);
output N6,N7;
wire N9,N12,N13,N14,N3_NOT,N4_NOT,EX1,EX2,EX3,EX4,EX5,EX6,EX7,EX8,EX9,
EX10,EX11,N2;
			nand2 gate1( .a(N1), .b(N3), .O(N14) );
			inv1 gate( .a(N3),.O(N3_NOT) );
			inv1 gate( .a(N4),.O(N4_NOT));
			and2 gate( .a(N3_NOT), .b(p1), .O(EX2) );
			and2 gate( .a(N4_NOT), .b(EX2), .O(EX3) );
			and2 gate( .a(N3), .b(p2), .O(EX4) );
			and2 gate( .a(N4_NOT), .b(EX4), .O(EX5) );
			and2 gate( .a(N3_NOT), .b(p3), .O(EX6) );
			and2 gate( .a(N4), .b(EX6), .O(EX7) );
			and2 gate( .a(N3), .b(p4), .O(EX8) );
			and2 gate( .a(N4), .b(EX8), .O(EX9) );
			or2  gate( .a(EX3), .b(EX5), .O(EX10) );
			or2  gate( .a(EX7), .b(EX10), .O(EX11) );
			or2  gate( .a(EX9), .b(EX11), .O(N9) );
			nand2 gate3( .a(N2), .b(N9), .O(N13) );
			nand2 gate4( .a(N5), .b(N9), .O(N12) );
			nand2 gate5( .a(N14), .b(N13), .O(N6) );
			nand2 gate6( .a(N12), .b(N12), .O(N7) );
			xor2 gate( .a(X_1), .b(N2_new), .O(N2) );
			endmodule
  		\end{verbatim}
	\end{tabular}
	\newline
	There are several points need attention:
	\begin{itemize}
		\item format is based on ABC format.
		\item primary input should be noted by \///RE\_\_PI.
		\item control bits should be noted by  \///RE\_\_ALLOW.
		\item allowed bits for each group of control bits should be written after  \///RE\_\_ALLOW, for example  \///RE\_\_ALLOW(1,0). If one camouflage gate requires more than one bit, for example it needs two bits, use the format  \///RE\_\_ALLOW(00,01,10,11) alternatively. 
	\end{itemize}
	\subsection{Call Tree}
	\tikzstyle{decision} = [diamond, draw, fill=blue!20, 
  			  text width=4.5em, text badly centered, node distance=3cm, inner sep=0pt]
	\tikzstyle{block} = [rectangle, draw, fill=blue!20, 
   			 text width=7em, text centered, rounded corners, minimum height=4em]
	\tikzstyle{line} = [draw, -latex']
	\tikzstyle{cloud} = [draw, ellipse,fill=red!20, node distance=3cm,
    				minimum height=2em]
	\begin{tikzpicture}[node distance = 3cm, auto]
		\node[block](init){translate and build milter};
		\node[block, below of = init](solver){solve to find a PI to differentiate two control bits};
		\node[decision, below of = solver](decide){solvable?};
		\node[block, left of = decide](find){find corresponding PO};
		\node[block, above of = find](duplicate){duplicate camouflage circuit and assign newly found PIPO};
		\node[block, right of = decide](finalSolu){based on collected PIPOs solve for final CB};
		
		\path[line](init) -- (solver);
		\path[line](solver) -- (decide);
		\path[line](decide) --node{Yes} (find);
		\path[line](find) -- (duplicate);
		\path[line](duplicate) -- (solver);
		\path[line](decide) -- node{No}(finalSolu);
		
	\end{tikzpicture}
\section{History Version}
	\begin{itemize}
		\item version 0.0.0, 04 Nov 2015
	\end{itemize}
\section{Contact}
\section{Citation}
	\begin{itemize}
		\item\emph{Oracle-Guided Incremental SAT Solving to Reverse
         Engineer Camouflaged Logic Circuits}
	\end{itemize}		





\end{document}